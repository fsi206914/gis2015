% Created 2015-03-04 Wed 08:54
\documentclass[11pt]{article}
\usepackage[utf8]{inputenc}
\usepackage[T1]{fontenc}
\usepackage{fixltx2e}
\usepackage{graphicx}
\usepackage{longtable}
\usepackage{float}
\usepackage{wrapfig}
\usepackage{rotating}
\usepackage[normalem]{ulem}
\usepackage{amsmath}
\usepackage{textcomp}
\usepackage{marvosym}
\usepackage{wasysym}
\usepackage{amssymb}
\usepackage{hyperref}
\tolerance=1000
\usepackage{tikz}
\usepackage{marginnote}
\usepackage{microtype}
\usepackage[inner=2.5cm,outer=6cm,marginparwidth=3.5cm, marginparsep=.5cm]{geometry}
\usetikzlibrary{shapes,arrows,positioning}
\usepackage[backend=bibtex, style=numeric]{biblatex}
\addbibresource{/home/gongzhitaao/Documents/gis2015/ref.bib}
\renewcommand*{\marginfont}{\footnotesize}

\hypersetup{
    bookmarks=true,          % show bookmarks bar?
    unicode=true,            % non-Latin characters in Acrobat's bookmarks
    pdftoolbar=true,         % show Acrobat's toolbar?
    pdfmenubar=true,         % show Acrobat's menu?
    pdffitwindow=false,      % window fit to page when opened
    pdfstartview={FitW},     % fits the width of the page to the window
    % pdftitle={},             % title
    pdfauthor={Zhitao Gong}, % author
    % pdfsubject={},           % subject of the document
    % pdfcreator={},           % creator of the document
    % pdfproducer={},          % producer of the document
    % pdfkeywords={keyword1} {key2} {key3}, % list of keywords
    pdfnewwindow=true,       % links in new PDF window
    colorlinks=true,         % false: boxed links; true: colored links
    linkcolor=red,           % color of internal links (change box color with linkbordercolor)
    citecolor=green,         % color of links to bibliography
    filecolor=magenta,       % color of file links
    urlcolor=cyan            % color of external links
}

\newcommand{\inlinelatex}[1]{#1}

\author{Zhitao Gong}
\date{\today}
\title{SIGSPATIAL CUP 2015}
\hypersetup{
  pdfkeywords={},
  pdfsubject={},
  pdfcreator={Emacs 24.3.1 (Org mode 8.2.10)}}
\begin{document}

\maketitle
\begin{abstract}
This draft summarizes the problem, restrictions, concerns and possible
solutions for SIGSPATIAL CUP 2015.
\end{abstract}

\section{Problem}
\label{sec-1}

In route planning one has to consider various restrictions like
turns, speed restrictions, one-way streets, etc.  In this cup, in
addition to the commonly used turn and speed restrictions, the
solution to the shortest path implementation should also satisfy the
polygonal obstacles.  Obstacles will be defined in terms of \(k\)
\emph{convex} polygons.  Shortest paths should avoid crossing these
polygons.

\begin{description}
\item[{Input}] A road network, a set of interconnecting lines and points
that represent a system of roads in a given region.
\item[{Output}] Shortest path, based on distance or time between source
and destination nodes and satisfies a given set of constraints.
\item[{Constraints}] Various constraints are applied,
\begin{enumerate}
\item Should obey turn restrictions where available,
\item speed restrictions, and
\item Should obey polygonal obstacles.
\end{enumerate}
\end{description}

\section{Restrictions}
\label{sec-2}

In this section we review the restrictions and our draft solutions
to each restriction.

\subsection{Turn Restriction}
\label{sec-2-1}

A turn restriction at a junction is not limited to turns, but can
also be used for instance if you are only allowed to go straight
on \cite{osm:wiki:restriction}.

Directed graph may not be a valid solution for this.  Consider the
following case shown in Figure \ref{fig:turn}.

\begin{figure}[ht]
\label{fig:turn}
\centering
\begin{tikzpicture}
  \tikzstyle{vertex} = [draw,circle,fill=gray,inner sep=2pt]
  \tikzstyle{cross} = [cross out,draw,minimum
  size=2*(#1-\pgflinewidth), inner sep=0pt, outer sep=0pt]

  \node [vertex,label={above:A}] at (-3, 0) (a) {};
  \node [vertex,label={above:B}] at (0, 0) (b) {};
  \node [vertex,label={above:C}] at (3, 0) (c) {};
  \node [vertex,label={right:D}] at (0, -3) (d) {};

  \draw (a) -- (b) -- (c); \draw (b) -- (d);

  \path [draw,->,>=stealth',rounded corners]
  ([yshift=3em,xshift=-2em]d.north west) |-
  ([yshift=-2em,xshift=2em]a.south east);

  \path [draw,->,>=stealth'] ([yshift=2em,xshift=-3em]c.north west) --
  ([yshift=2em,xshift=3em]a.north east);

  \draw (-.9,-.9) node[cross,red,minimum size=1em] {};
\end{tikzpicture}

\caption{Turn restriction}\label{fig:turn}
\end{figure}

Traffic coming from \(D\) is not allowed to turn right at \(B\).
In a directed graph representation, the edge \(B\to A\) does not
exist.  However traffic coming from \(C\) is allowed to go straight
on at \(B\).  Thus, there should an edge \(B\to A\), which is
contradictory.

To solve this, we consider using the undirected graph.  But
globally, we keep a hash map containing all the turn restrictions
\marginnote{Suggested by liang, to make it more space efficient}.
For example, the information we keep for Figure \ref{fig:turn} is
\(\{\langle D, B\rangle\to \{A\}\}\), with index being an \emph{ordered}
pair, e.g. \(\langle D, B\rangle\), and value being an \emph{unordered}
set containing all the restricted turns, e.g. \(\{A\}\).  During a
graph traversal, we need to keep the previous vertex, which is
trivial in implementation.

\subsection{Shortest Distance and Time}
\label{sec-2-2}

Although the problem specify two metrics, distance and time, they
make no difference in implementation.  Both are edge weights.  Each
road segment has speed limit.  Road segment length divided by its
speed limit yields the shortest time for this segment.  We can
store both weights in each edge.

\subsection{Polygon Obstacle}
\label{sec-2-3}

There exist \emph{convex} polygon obstacles and no path should go
through any of these.  This is the tricky part.  However there is a
strong assumption that the polygons are \emph{convex}, which makes the
problem much easier.  For this restrictions, a straightforward
solution is to remove any vertices and edges attached to them
inside a polygon obstacle.  This is, however, time-consuming.  For
a single point and a polygon, the complexity of Point-in Polygon
(PIP) check is linear, \(O(n)\), where \(n\) is the number of
vertices in a polygon.  This will give us the optimal solution with
any shortest path algorithm.  A more aggressive solution would be
removing points inside the isothetic minimum bounding rectangles of
these polygons.  Then for a single point and a polygon, the PIP
test is constant.  This may yields sub-optimal solution, however
this will speed up the program significantly.  Further conclusion
about the accuracy and speedup will await until our benchmark.

\section{Shortest Path Algorithm}
\label{sec-3}

Single source shortest path is a well studied problem in graph
theory.  Dijkstra's algorithm \cite{Dijkstra:1959}, conjured by
Dijkstra in 1959, is the earlier one within my knowledge.  And
Bellman et al. extended Dijkstra's algorithm to graph with negative
edge weights \cite{bellman:1958} around 1958.  And later in 1968,
Hart et al. published a general form of Dijkstra's algorithm, named
the \(A^*\) search which achieves better performance by using
\emph{heuristics}.  Please refer to the more comprehensive survey
\cite{Cherkassky:1996} on this topic.

Since we are not dealing with negative edge weights, here we only
consider Dijkstra's algorithm and its generalized form, \(A^*\).
And their only difference is the cost function during the search,
so we do not distinguish them and use the name Dijkstra's algorithm
in our discussion.

The complexity of original Dijkstra's algorithm without a
min-priority queue is \(O(|V|^2)\).  Using Fibonacci heap based
min-priority queue \cite{Fredman:1984} brings the complexity down
to \(O(|E| + |V|\log|V|)\).  This is asymptotically the fastest
known single-source shortest-path algorithm for arbitrary graph
with unbounded non-negative weights.  With more assumptions,
however, this complexity could be further reduced.  According to
\cite{Wikipedia:dijkstra}, when edge weights are integers and
bounded by a constant \(C\), the usage of a special priority queue
structure \cite{VanEmdeboas:1976} brings the complexity down to
\(O(|E|\log\log|C|)\).  And another implementation based on a
combination of a new radix heap and Fibonacci Heap runs in
\(O(|E| + |V|\sqrt{\log|V|})\) \cite{Ahuja:1990}.  The algorithm
given by \cite{Thorup:2000} runs in \(O(|E|\log\log|V|)\) and the
one given by \cite{Raman:1997} runs in
\(O(|E| + |V|\min{\left\{(\log|V|)^{1/3+\epsilon},
  (\log|C|)^{1/4+\epsilon}\right\}}\).  And finally
\cite{Thorup:1999} shows the Dijkstra's algorithm can be completed
in linear time \(O(|V|+|E|)\).

The method \cite{Thorup:1999} is enticing.  We can multiply the all
the weights by 1000 and leave out the remaining after the decimal
point.  This may give us some rounding error and result in
sub-optimal solution, however the running time is unbeatable.

\section{Miscellaneous}
\label{sec-4}

The data are stored in shapefile \cite{Wikipedia:shapefile}.
Although there is an open source library \cite{Warmerdam}, we may
not need to parse all the information.  So we need a simplified and
hopefully faster version to extract all relevant information while
skipping all information we do not need for processing.

\small\printbibliography
% Emacs 24.3.1 (Org mode 8.2.10)
\end{document}
